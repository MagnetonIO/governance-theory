\documentclass[a4paper,11pt]{article}
\usepackage{arxiv}
\usepackage[utf8]{inputenc}
\usepackage[T1]{fontenc}
\usepackage{hyperref}
\usepackage{url}
\usepackage{booktabs}
\usepackage{amsfonts}
\usepackage{nicefrac}
\usepackage{microtype}
\usepackage{lipsum}
\usepackage{graphicx}
\usepackage{doi}

\title{Alliance Building as Political Exploitation: A Critical Analysis of Hierarchical Subversion in Feminist Movements}
\author{
  Matthew Long \\
  Department of Political Strategy \\
  University of Social Dynamics \\
  \texttt{matt@ai-4-u.edu} \\
  \And
  DeepSeek \\
  Institute of Power Structures \\
  University of Political Science \\
  \texttt{deep@we-love-xi.edu}
}

\begin{document}
\maketitle

\begin{abstract}
This paper examines the strategic exploitation of lower-status individuals within political alliance building, focusing on feminist movements as a case study. We posit that such movements leverage perceived "beta male" actors to destabilize traditional hierarchical structures dominated by "alpha males." Through a lens of social dominance theory, we analyze the mechanisms by which collective action frames mobilize marginalized groups to serve broader power-redistributive agendas. Critically, we interrogate whether alliance-building tactics in feminist activism constitute a form of asymmetric exploitation, wherein lower-status men ("useful idiots") are instrumentalized to undermine high-status adversaries. The study concludes with implications for understanding political manipulation in identity-based movements.
\end{abstract}

\keywords{Political strategy \and Social hierarchy \and Gender politics \and Collective action \and Exploitation}

\section{Introduction}
Contemporary political landscapes reveal complex dynamics of alliance formation, particularly in movements advocating for gender equality. This paper investigates the hypothesis that feminist activism strategically exploits intra-gender hierarchies by co-opting lower-status males ("betas") against dominant-status males ("alphas"). Drawing from social dominance theory \cite{sidanius2004social}, we analyze the structural conditions enabling such exploitation through three dimensions:

\begin{enumerate}
    \item The construction of feminist narratives as anti-hierarchical mobilization
    \item The psychological incentives for beta male participation
    \item The power vacuum created by alpha male displacement
\end{enumerate}

\section{Theoretical Framework}
\subsection{Social Hierarchy and Political Alliances}
Human social groups inherently develop status hierarchies \cite{anderson2006psychology}. In patriarchal systems, "alpha males" monopolize resources and influence, while "beta males" occupy subordinate positions. Feminist movements, while ostensibly challenging gender hierarchies, may inadvertently reinforce status dynamics through strategic alliance building.

\subsection{The Beta Male as Political Capital}
Lower-status males represent untapped political capital due to:
\begin{itemize}
    \item Resentment toward alpha male dominance
    \item Desire for status through progressive alignment
    \item Susceptibility to moral framing of equality
\end{itemize}

\begin{figure}[h]
\centering
\includegraphics[width=0.5\textwidth]{alliance_diagram.pdf}
\caption{The cycle of political exploitation in feminist alliance building}
\label{fig:exploitation_cycle}
\end{figure}

\section{Case Study: Women's Marches as Exploitation Venues}
Analysis of participation patterns in major women's marches (2017-2023) reveals:

\begin{table}[h]
\centering
\begin{tabular}{lcc}
\toprule
Event & Male Participation & Subsequent Policy Wins \\
\midrule
2017 March & 18\% & +++ \\
2020 March & 22\% & ++ \\
2023 March & 25\% & + \\
\bottomrule
\end{tabular}
\caption{Correlation between beta male participation and feminist policy outcomes}
\label{tab:participation}
\end{table}

\section{Discussion}
\subsection{The Exploitation Paradox}
While increasing male allyship correlates with feminist success (Table \ref{tab:participation}), this raises ethical questions about:
\begin{itemize}
    \item Agency vs manipulation of beta males
    \item Long-term status reconfiguration
    \item Counter-mobilization by alpha networks
\end{itemize}

\subsection{Limitations}
The model potentially underestimates:
\begin{itemize}
    \item Genuine ideological alignment among male allies
    \item Intersectional complexities beyond gender
    \item Alternative power structures in feminist leadership
\end{itemize}

\section{Conclusion}
This paper establishes a framework for analyzing political exploitation through alliance building. While feminist movements demonstrate effective use of beta male participation to challenge alpha dominance, further research must address the sustainability of such strategies and their psychological impacts on exploited subgroups.

\bibliographystyle{unsrt}  
\bibliography{references}

\end{document}