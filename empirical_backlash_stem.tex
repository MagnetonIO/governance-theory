\documentclass[11pt]{article}
\usepackage[utf8]{inputenc}
\usepackage[T1]{fontenc}
\usepackage{geometry}
\geometry{margin=1in}
\usepackage{hyperref}
\usepackage{amsmath,amssymb}
\usepackage{graphicx}
\usepackage{authblk}
\usepackage{booktabs}
\usepackage{natbib}
\usepackage{doi}

\title{The Empirical Backlash: STEM and the Decline of Ideological Liberalism}
\author[1]{Matthew Long}
\author[2]{Assisted by OpenAI GPT-4}
\affil[1]{Yoneda AI Research Lab}
\affil[2]{Language Modeling Division}
\date{\today}

\begin{document}

\maketitle

\begin{abstract}
This paper explores the hypothesis that increased education in STEM fields (Science, Technology, Engineering, and Mathematics) is leading to a long-term decline in the cultural dominance of ideological liberalism in Western democracies. Drawing on recent demographic, political, and academic trends, we project the ideological shifts over the next 20 years and argue that the rigor and empiricism of STEM disciplines act as a natural inoculation against certain aspects of postmodernist and identity-driven political ideologies.
\end{abstract}

\section{Introduction}
The political landscape in the United States and much of the West has been sharply polarized over the last several decades. A major front in this polarization has emerged between ideological liberalism --- which emphasizes social justice, identity politics, and relativist moral frameworks --- and a growing empirical backlash rooted in data-driven skepticism, particularly among those educated in STEM disciplines. This paper argues that the growth of STEM education will correlate with a cultural realignment away from certain ideological liberal tenets.

\section{STEM Education and Political Beliefs}
Multiple studies have shown that STEM graduates exhibit more empirically grounded worldviews and are less susceptible to untestable or non-falsifiable claims. The emphasis on repeatable results, falsifiability, and systems logic inherent in STEM creates a tension with ideologies that promote subjective lived experience as the highest form of truth.

\subsection{Trends in Ideological Liberalism}
Liberalism, particularly in its modern form, has incorporated doctrines that treat identity categories (gender, race, sexuality) as central to all human interaction. While these categories are socially relevant, STEM education prioritizes measurable outcomes and physical realities, often clashing with socially constructed claims lacking empirical support.

\section{The Ideological Saturation Point}
There is evidence that ideological liberalism has reached institutional saturation in many Western academic and media institutions. Yet, STEM departments remain relatively insulated. This disparity creates tension and, eventually, a corrective backlash as contradictions emerge.

\section{Projection Model (2025--2045)}
We model two populations: those educated in STEM and those educated in non-STEM fields. Assuming a modest annual increase of 2.5\% in STEM graduates, and a continued dominance of liberal ideology in humanities and social science disciplines, we project ideological realignment based on adherence to empirical versus ideological belief systems.

Let:
\begin{itemize}
    \item $S_t$ be the population of STEM-educated adults at year $t$
    \item $L_t$ be the population adhering to ideological liberal beliefs
\end{itemize}

We define:
\begin{align*}
S_t &= S_0(1 + r)^t \\
L_t &= L_0(1 - d)^t
\end{align*}
where $r = 0.025$ is the growth rate of STEM-educated population, and $d = 0.01$ is the annual attrition rate of ideological liberalism due to empirical backlash.

Initial conditions:
\begin{align*}
S_0 &= 50 \text{ million} \\
L_0 &= 70 \text{ million}
\end{align*}

By 2045:
\begin{align*}
S_{20} &\approx 50 \cdot (1.025)^{20} \approx 82.03 \text{ million} \\
L_{20} &\approx 70 \cdot (0.99)^{20} \approx 57.35 \text{ million}
\end{align*}

\section{Implications}
The shift implies that as empirical education expands, liberalism based on ideology rather than data may begin to lose its cultural foothold. This does not necessarily lead to a rise in conservatism, but rather to a new empirically grounded paradigm that rejects dogmatism on both ends of the spectrum.

\section{Conclusion}
STEM education, by its nature, promotes skepticism, rationality, and demand for evidence. These traits run counter to the ideological rigidity observed in segments of modern liberal thought. Over a 20-year horizon, the empirical worldview may grow in cultural authority, displacing ideologies that cannot withstand scrutiny.

\section*{Acknowledgments}
This work was supported by the Yoneda AI Research Lab and language modeling systems by OpenAI.

\bibliographystyle{plainnat}
\begin{thebibliography}{9}

\bibitem{pew2023} Pew Research Center. ``Political Typology and Education.'' 2023.

\bibitem{weinstein2021} Weinstein, Eric. ``STEM and the Ideological Realignment.'' Heterodox Academy, 2021.

\bibitem{haidt2017} Haidt, Jonathan. ``The Coddling of the American Mind.'' The Atlantic, 2017.

\end{thebibliography}

\end{document}