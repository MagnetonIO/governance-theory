\documentclass[11pt]{article}
\usepackage{arxiv}
\usepackage[utf8]{inputenc}
\usepackage[T1]{fontenc}
\usepackage{hyperref}
\usepackage{url}
\usepackage{booktabs}
\usepackage{amsfonts}
\usepackage{nicefrac}
\usepackage{microtype}
\usepackage{graphicx}
\usepackage{cleveref}
\usepackage{authblk}

\title{The Ideological Capture of Western Ivy League Universities and the End of an Era: The Rise of the Right and Trumpism Globally}

\author[1]{Matthew Long}
\author[2]{DeepSeek}

\affil[1]{Independent Researcher \\ \texttt{matthewlong@example.com}}
\affil[2]{DeepSeek Research \\ \texttt{research@deepseek.com}}

\begin{document}

\maketitle

\begin{abstract}
    This paper examines the ideological homogeneity within Western Ivy League universities, arguing that institutional bias toward progressive-left ideology has led to a decline in intellectual diversity. We analyze the mechanisms of ideological capture—including hiring practices, curriculum design, and institutional incentives—that have reinforced this trend. Further, we explore the political and cultural backlash against this hegemony, exemplified by the rise of right-wing populism, the election of Donald Trump in the U.S., and similar movements globally. We conclude that the monopoly of progressive ideology in elite academia is fracturing due to external political shifts and growing public skepticism, marking the end of an era in higher education.
\end{abstract}

\keywords{Ideological capture, Ivy League universities, Trumpism, Right-wing populism, Higher education}

\section{Introduction}
For decades, elite Western universities—particularly the Ivy League—have been perceived as bastions of progressive-left ideology. Critics argue that these institutions have systematically marginalized conservative, libertarian, and heterodox viewpoints, creating an environment of ideological conformity. This paper investigates the mechanisms of this ideological capture and its consequences, including the societal backlash that has fueled the rise of right-wing movements, such as Trumpism in the U.S. and parallel movements in Europe and beyond.

\section{The Mechanisms of Ideological Capture}
\subsection{Hiring and Tenure Practices}
Studies indicate that faculty in Ivy League institutions lean overwhelmingly progressive, with conservatives and right-leaning academics underrepresented \cite{Gross2016}. Hiring committees often prioritize ideological alignment over intellectual diversity, reinforcing homogeneity.

\subsection{Curriculum and Pedagogy}
The dominance of critical theory, postmodernism, and identity-based frameworks in humanities and social sciences has reshaped curricula. Dissenting perspectives are frequently excluded or stigmatized, narrowing the range of acceptable discourse.

\subsection{Institutional Incentives}
Universities reward conformity through funding, promotions, and peer validation. Heterodox scholars face professional penalties, discouraging ideological diversity.

\section{The Backlash: Rise of the Right and Trumpism}
\subsection{Public Distrust in Academia}
Polls show declining public confidence in higher education, with many perceiving universities as politically biased \cite{Pew2019}. This skepticism has been exploited by right-wing populists.

\subsection{Trump and the Populist Revolt}
Donald Trump’s 2016 election symbolized a rejection of elite institutions, including academia. His rhetoric against "coastal elites" resonated with voters alienated by the perceived hegemony of progressive ideology.

\subsection{Global Parallels}
Similar movements—such as Brexit, the AfD in Germany, and Bolsonarism in Brazil—reflect a broader rejection of progressive institutional dominance.

\section{The End of an Era?}
The ideological monopoly of elite academia is under threat. Alternative educational platforms (e.g., online forums, conservative colleges) are gaining traction. Political shifts may force universities to re-embrace intellectual pluralism or risk irrelevance.

\section{Conclusion}
The ideological capture of Ivy League universities has contributed to a cultural and political rift, accelerating the rise of right-wing populism. As backlash grows, the progressive hegemony in higher education may be unsustainable, heralding a new era of ideological competition.

\bibliographystyle{unsrt}
\bibliography{references}

\end{document}