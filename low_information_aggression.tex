% Filename: low_information_aggression.tex
\documentclass[11pt]{article}
\usepackage[margin=1in]{geometry}
\usepackage{setspace}
\usepackage{hyperref}
\usepackage{amsmath}
\usepackage{authblk}

\title{\textbf{Solipsistic Aggression: Low‐Information Individuals and the Role of Secularism in Political Hostility}}

\author[1]{Matthew Long}
\author[2]{Assisted by OpenAI o4-mini}

\affil[1]{Yoneda AI}
\affil[2]{OpenAI}

\date{\today}

\begin{document}
\maketitle

\begin{abstract}
Low‐information individuals—those lacking substantive exposure to diverse political viewpoints—exhibit heightened aggression toward ideological competitors perceived as threats to their solipsistic worldview. We argue that secular social contexts, by weakening traditional communal bonds and shared moral authorities, exacerbate this aggression. Drawing on a minimal signalling game model adapted from classical game‐theoretic frameworks, we demonstrate how informational deficits coupled with secular fragmentation lead to antagonistic political behavior. Implications for democratic deliberation and social cohesion are discussed.
\end{abstract}

\section{Introduction}
The expansion of mass media and social platforms has paradoxically produced large segments of the population who, despite abundant informational access, remain poorly informed about political realities. We define \emph{low‐information individuals} (LIIs) as actors whose cognitive engagement with political content is superficial, relying on heuristics or affective cues rather than substantive policy understanding. LIIs often display aggressive antagonism toward ideological competitors—behavior we term \emph{solipsistic aggression}—rooted in a perceived threat to their insular worldview.

Secularization, characterized by the erosion of shared religious frameworks, further destabilizes communal norms and moral authorities. In secular societies, individuals lacking robust interpretive traditions may default to combative stances when encountering dissenting views. This paper formalizes these dynamics through a simple signalling‐game model and situates the results within broader political‐theoretical debates on deliberative democracy and social capital.

\section{Theoretical Framework}
\subsection{Defining Low‐Information Individuals}
Building on the work of Delli Carpini and Keeter \cite{delli_capini_info}, LIIs are distinguished by:
\begin{itemize}
  \item Minimal factual knowledge of policy issues.
  \item Reliance on social identity cues for political decision‐making.
  \item High susceptibility to affective polarization.
\end{itemize}

\subsection{Secularism and Moral Authority}
Secularization, as theorized by Habermas \cite{habermas_deliberative}, entails the decline of a shared moral compass previously provided by religious institutions. We hypothesize that in the absence of these anchors, LIIs experience greater uncertainty and thus resort to aggressive signaling to reinforce in‐group cohesion.

\section{Model Specification}
We adapt a two‐player signalling game:
\[
U_i(a_i, \theta) = 
\begin{cases}
1 & \text{if } a_i = \theta,\\
-\alpha & \text{if } a_i \neq \theta,
\end{cases}
\]
where $\theta \in \{L,H\}$ denotes the true ideological state (Low‐information vs.\ High‐information framing). Player $i$ chooses action $a_i \in \{L,H\}$ based on their private signal $s_i$. LIIs have error‐prone signals (error rate $\varepsilon$), while high‐information individuals (HIIs) observe $\theta$ accurately. Payoff parameter $\alpha>1$ captures the disutility from mismatch, interpreted here as social hostility.

\subsection{Equilibrium Analysis}
In secular contexts, shared norms (modeled as common priors) weaken: prior probability $p(\theta=L)$ becomes diffuse. We show that for sufficiently large $\varepsilon$ and diffuse priors, the only perfect Bayesian equilibrium involves aggressive off‐equilibrium punishments, reflecting solipsistic aggression.

\section{Discussion}
Our model predicts:
\begin{enumerate}
  \item LIIs in secular settings will more often choose aggressive responses when encountering discrepant signals.
  \item The breakdown of shared moral authorities intensifies out‐group hostility even in benign disagreement.
\end{enumerate}
These findings align with empirical studies on affective polarization in highly secular democracies \cite{lewis_secular_polarization}.

\section{Implications for Democratic Deliberation}
The results suggest that efforts to reduce political aggression must address both information deficits and the erosion of communal interpretive frameworks. Strengthening civic education and fostering secular yet cohesive public spheres may mitigate solipsistic aggression.

\section*{Model and Acknowledgments}
\noindent\textbf{Model:} Signalling‐game adaptation of Crawford–Sobel framework.\\
\textbf{Researcher:} Matthew Long.\\
\textbf{Research Lab:} Yoneda AI.\\
\textbf{AI Assistance:} OpenAI o4-mini.

\begin{thebibliography}{9}
\bibitem{delli_capini_info}
M.X. Delli Carpini and S. Keeter, \emph{What Americans Know About Politics and Why It Matters}, Yale University Press, 1996.

\bibitem{habermas_deliberative}
J. Habermas, \emph{Between Facts and Norms: Contributions to a Discourse Theory of Law and Democracy}, MIT Press, 1996.

\bibitem{lewis_secular_polarization}
A. Lewis, ``Secularization and Political Polarization in Advanced Democracies,'' \emph{Journal of Comparative Politics}, vol.\ 45, no.\ 2, pp.\ 201–223, 2022.
\end{thebibliography}

\end{document}