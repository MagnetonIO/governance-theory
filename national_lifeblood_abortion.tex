\documentclass[11pt]{article}
\usepackage[margin=1in]{geometry}
\usepackage{amsmath, amssymb}
\usepackage{graphicx}
\usepackage{hyperref}
\usepackage{authblk}
\usepackage{setspace}
\setstretch{1.2}

\title{\textbf{The Nation's Lifeblood: How Population Decline and Universal Abortion Policies Threaten National Continuity and Security}}
\author[1]{Matthew Long}
\author[2]{Assisted by Yoneda AI Research}
\affil[1]{Independent National Strategy Analyst}
\affil[2]{AI-Augmented Population Resilience Lab}
\date{\today}

\begin{document}

\maketitle

\begin{abstract}
A nation's strength resides in its people. While debates around abortion often center on individual rights and social values, this paper reframes the issue within the context of national security and demographic resilience. We argue that widespread, unrestricted abortion policies—when combined with aging populations, falling birthrates, and increasing reliance on immigration—pose existential risks to civilizational continuity. Drawing on demographic modeling and fertility rate projections, we call for a recalibration of policy frameworks to balance reproductive freedom with strategic population sustainability.
\end{abstract}

\section{Introduction: Population as Strategic Infrastructure}
Nations rise and fall not merely on material resources or military capacity, but on their ability to sustain, reproduce, and integrate generational human capital. A healthy demographic pyramid is essential for maintaining:
\begin{itemize}
    \item Economic productivity
    \item Military viability
    \item Cultural continuity
    \item Institutional memory
\end{itemize}

\section{Demographic Collapse and Western Fertility Trends}
\subsection{Replacement Rate Threshold}
A fertility rate (TFR) of 2.1 children per woman is needed for population stability. As of 2023:
\begin{itemize}
    \item U.S. TFR = 1.66
    \item Germany = 1.51
    \item Japan = 1.26
\end{itemize}

\subsection{Modeling Population Decay}
Let $P(t)$ be population at time $t$ and $r$ be the net reproduction rate:
\begin{equation}
P(t) = P_0 e^{rt}, \quad r < 0 
\end{equation}
Under persistent below-replacement fertility, the population undergoes exponential decay.

\section{Abortion Policy and Demographic Stress}
\subsection{Quantifying Impact}
Let $A(t)$ be the annual abortion count. Cumulative demographic suppression $\Delta P$ over $T$ years is:
\begin{equation}
\Delta P = \int_0^T A(t) e^{r(T-t)} dt
\end{equation}
This formula accounts for the future reproductive impact of terminated pregnancies.

\subsection{Strategic Considerations}
\begin{itemize}
    \item Reduced future labor force entrants
    \item Loss of potential innovators, servicemembers, and caregivers
    \item Increased dependency ratio stress on existing workers
\end{itemize}

\section{Abortion as National Security Variable}
\subsection{Military Readiness Forecasting}
Let $M(t)$ be military-eligible population:
\begin{equation}
M(t) = \beta P(t-18), \quad \beta \in [0.05, 0.1]
\end{equation}
Where $\beta$ captures health, aptitude, and willingness thresholds. Lower fertility and higher abortion rates compress $M(t)$ and erode readiness.

\subsection{Geo-strategic Vulnerabilities}
Nations with declining populations:
\begin{itemize}
    \item Depend on migration to sustain industries
    \item Suffer cultural fragmentation and low cohesion
    \item Lose global leverage relative to expanding powers (e.g., China, India)
\end{itemize}

\section{Toward a Balanced Population Policy}
\subsection{Pro-Natal Incentives and Reproductive Ethics}
\begin{itemize}
    \item Paid parental leave and tax relief for families
    \item Adoption and family growth support infrastructure
    \item Encouraging life-affirming narratives and responsibilities
\end{itemize}

\subsection{Reevaluating Abortion as a Public Policy Tool}
\begin{itemize}
    \item Abortion not as civil right but as demographic cost
    \item Introduce proportional impact analysis in state reporting
    \item Tie reproductive policy into long-term security and economic planning
\end{itemize}

\section{Conclusion: The Stakes of Reproduction}
Nations cannot persist without people. Universal abortion access—when removed from demographic planning—becomes not only a moral question but a strategic miscalculation. Sustainable sovereignty demands population foresight, cultural confidence, and structural investment in generational continuity.

\end{document}